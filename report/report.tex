\documentclass[12pt]{article}
\usepackage{amsmath, amsfonts, amssymb}
\usepackage{mathtools}
\usepackage{datetime}
\usepackage{interval}

\DeclarePairedDelimiter\floor{\lfloor}{\rfloor}
\setlength{\parskip}{1.5em}
\title{MATH3320 Project Report\\Topic: Image Compression}
\author{ZHANG Xinfang 1155141566\\ZHANG Yifan 1155141570}
\newdate{date}{08}{11}{2021}
\date{\displaydate{date}}


\begin{document}
\maketitle
\newpage

% Part 1: introduction

\section*{1.\quad Introduction}
\begin{flushleft}
In recent years, with the rapid developement in technology, multimedia product 
of digital information grows increasingly fast, which requires a large memory space 
and sufficient bandwidth in the storage and transmission process. 
Therefore, data compression becomes extremely vital for reducing the data 
redundancy to save more hardware space and transmission bandwidth. 

Image compression is the process of removing redundant and irrelevant information, 
and efficiently encoding or reverting what remains without affecting or degrading its quality. 
The objective of image compression is to store or transmit data in an efficient form
and to reduce the storage quantity as much as possible.

One useful techniques in image compression is to decompose an image into linear combination 
of elementary images with specific properties. By truncating some less important components 
in the image decomposition, we can compress an image to reduce the image size 
and achieve transform coding. 

In this paper, we will discuss some useful image decomposition methods, demonstrate the 
applications of these decomsotion methods for image compression and analyze their advantages,
disadvantages and applicability.
\end{flushleft}

% Part2: Method
    % 1) definition + formula
    % 2) proof + error computation
    % 3) code + example

\section*{2.\quad Image Decomposition methods}
\subsection*{2.1\quad Singular Value Decomposition (SVD)}
For 
Singular Value Decomposition is a method 

\subsection*{2.2\quad Haar Transform}


\subsection*{2.3\quad Walsh Transform}


\subsection*{2.4\quad Discrete Fourier Transform (DFT)}


\subsection*{2.5\quad Even Discrete Cosine Transform (EDCT)}



% Part3: Result
    % Error, computation cost
    % (Ppt comparison chart???)
    % 說一說 為什麼? 哪種圖片更適合哪種方法? 方法的特點
    % dont ask me why, dont wanna open the pages again
    
\section*{3.\quad Results}





\end{document}